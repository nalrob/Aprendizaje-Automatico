% Template for CRC journal article

%-----------------------------------------------------------------------------------

%-----------------------------------------------------------------------------------

%%%%%%%%%%%%%%%%%%%%%%%%%%%%%%%%%%%%%%%%%%%%%%
%%%%%%%%%%%%%%%%%%%%%%%%%%%%%%%%%%%%%%%%%%%%%%
%%                                          %%
%% Important note on usage                  %%
%% -----------------------                  %%
%% This file must be compiled with PDFLaTeX %%
%% Using standard LaTeX will not work!      %%
%%                                          %%
%%%%%%%%%%%%%%%%%%%%%%%%%%%%%%%%%%%%%%%%%%%%%%
%%%%%%%%%%%%%%%%%%%%%%%%%%%%%%%%%%%%%%%%%%%%%%

%% The '5p' and 'times' class options of elsarticle are used
\documentclass[5p,times,authoryear]{sciarticle}

%% The `crc' package must be called to make the CRC functionality available
%% crc_RIAI es el paquete ecrc para la revista RIAI
\usepackage{crc_SCI}

%% The crc package defines commands needed for running heads and logos.
%% For running heads, you can set the journal name, the volume, the starting page and the authors

%%%%%%%%%%%%%%%%%%%%%%%%%%%%%%%%% Añadido por Secretaría RIAI
\usepackage[spanish]{babel}     % Idioma
\addto\captionsspanish{%
\def\tablename{Tabla}%
}
\usepackage[utf8]{inputenc}     % Lengua latina
\usepackage{amsmath}            % Para las referencias a ecuaciones con \eqref
\usepackage{epstopdf}           % Para poder insertar figuras .eps al compilar con PDFLATEX
\usepackage{flushend}           % Para igualar las columnas de la última página
%\usepackage{hyperref}          % Para hipervínculos dentro del PDF
%%%%%%%%%%%%%%%%%%%%%%%%%%%%%%%%%%%%%%%%%%%%%%%%%%%%%%%


%% set the volume if you know. Otherwise `00'
\volume{XVIII}

%% set the starting page if not 1
\firstpage{1}

%% Give the name of the journal
\journalname{Simposio CEA de Control Inteligente.}

%% Give the author list to appear in the running head
%% Example \runauth{C.V. Radhakrishnan et al.}
\runauth{Apellido primer autor, Inicial. et al. }

%% The choice of journal logo is determined by the \jid and \jnltitlelogo commands.
%% A user-supplied logo with the name <\jid>logo.pdf will be inserted if present.
%% e.g. if \jid{yspmi} the system will look for a file yspmilogo.pdf
%% Otherwise the content of \jnltitlelogo will be set between horizontal lines as a default logo

%% Give the abbreviation of the Journal. Contast the Publisher if in doubt what this is.
\jid{SCI}

%% Give a short journal name for the dummy logo (if needed)
\jnltitlelogo{}

%% Hereafter the template follows `elsarticle'.
%% For more details see the existing template files elsarticle-template-harv.tex and elsarticle-template-num.tex.

%% The conventions of elsarticle-template-num.tex should be followed (included below)
%% If using BibTeX, use the style file elsarticle-num.bst

%% End of ecrc-specific commands
%%%%%%%%%%%%%%%%%%%%%%%%%%%%%%%%%%%%%%%%%%%%%%%%%%%%%%%%%%%%%%%%%%%%%%%%%%

%% The amssymb package provides various useful mathematical symbols
\usepackage{amssymb}
%% The amsthm package provides extended theorem environments
%% \usepackage{amsthm}

%% The lineno packages adds line numbers. Start line numbering with
%% \begin{linenumbers}, end it with \end{linenumbers}. Or switch it on
%% for the whole article with \linenumbers after \end{frontmatter}.
%% \usepackage{lineno}

%% natbib.sty is loaded by default. However, natbib options can be
%% provided with \biboptions{...} command. Following options are
%% valid:

%%   round  -  round parentheses are used (default)
%%   square -  square brackets are used   [option]
%%   curly  -  curly braces are used      {option}
%%   angle  -  angle brackets are used    <option>
%%   semicolon  -  multiple citations separated by semi-colon
%%   colon  - same as semicolon, an earlier confusion
%%   comma  -  separated by comma
%%   numbers-  selects numerical citations
%%   super  -  numerical citations as superscripts
%%   sort   -  sorts multiple citations according to order in ref. list
%%   sort&compress   -  like sort, but also compresses numerical citations
%%   compress - compresses without sorting
%%
%% \biboptions{comma,round}

% \biboptions{}

% if you have landscape tables
\usepackage[figuresright]{rotating}

% put your own definitions here:
%   \newcommand{\cZ}{\cal{Z}}
%   \newtheorem{def}{Definition}[section]
%   ...

% add words to TeX's hyphenation exception list
%\hyphenation{author another created financial paper re-commend-ed Post-Script}

% para poder introducir varias figuras que ocupen el ancho de las dos columnas.
\usepackage{subfigure}
\usepackage{float} % JVS
\usepackage{multicol} % JVS
\usepackage[most]{tcolorbox}


% declarations for front matter

\begin{document}

\begin{frontmatter}

%% Title, authors and addresses
%% use the tnoteref command within \title for footnotes;
%% use the tnotetext command for the associated footnote;
%% use the fnref command within \author or \address for footnotes;
%% use the fntext command for the associated footnote;
%% use the corref command within \author for corresponding author footnotes;
%% use the cortext command for the associated footnote;
%% use the ead command for the email address,
%% and the form \ead[url] for the home page:
%% \title{Title\tnoteref{label1}}
%% \tnotetext[label1]{}
%% \author{Name\corref{cor1}\fnref{label2}}
%% \ead{email address}
%% \ead[url]{home page}
%% \fntext[label2]{}
%% \cortext[cor1]{}
%% \address{Address\fnref{label3}}
%% \fntext[label3]{}
\dochead{Predicción de la quema de calorias durante el ejercicio}
\datesite{19 de febrero de 2023}
%% Use \dochead if there is an article header, e.g. \dochead{Short communication}

% Escriba el título del artículo.
\title{Uso de técnicas de Machine Learning:\\
Kmeans + SVM}


%% use optional labels to link authors explicitly to addresses:
%% \author[label1,label2]{<author name>}
%% \address[label1]{<address>}
%% \address[label2]{<address>}

\author[First]{AUTOR: Robledo, Nallely.\corref{cor1}}


% Escriba el email del autor para correspondencia.
\cortext[cor1]{Autor para correspondencia: autor1@ceautomatica.es
\\
Attribution-NonCommercial-ShareAlike 4.0 International (CC BY-NC-SA 4.0)
}

\address[First]{Universidad Autonoma de Nuevo León, Facultad de Fisicomatematicas\\
Coordicación de maestria MCD , Pedro de Alba S/N, Niños Héroes, Ciudad Universitaria, San Nicolás de los Garza, N.L. }


% Escriba el nombre de los autores y el título de su artículo en inglés para que aparezca correctamente el cuadro "Como citar:"
\begin{comocitar}
\begin{center}
\begin{tcolorbox}[frame hidden, width=\linewidth*2, colframe=white, colback=black!8!white]
\tocitearticle{Robledo, Nallely. 2023. Calorie burn prediction during exercise. Universidad Autonoma de Nuevo León, Facultad de Ciencias Fisicomatematicas 00, 1-5. https://doi.org/10.4995/riai.2020.7133}
\end{tcolorbox}
\end{center}
\end{comocitar}

\begin{abstract}
%% Text of abstract
Si la gente respondiera honestamente a la pregunta '¿Cuáles son las razones por las que haces ejercicio?', una respuesta frecuente sería quemar calorías. De hecho, según el Departamento de Salud y Servicios Humanos de EE. UU. (1992), el 26 porciento de los adultos estadounidenses entre 20 y 74 años tienen sobrepeso, lo que demuestra claramente el impacto de esta preocupación nacional.

Se sabe que la reducción de la grasa corporal puede revertir varios procesos de enfermedades (p. ej., diabetes tipo II, enfermedades cardíacas, etc.), el ejercicio aumenta el gasto calórico total y también maximiza la pérdida de grasa corporal y el mantenimiento o aumento de masa muscular, la participación en el ejercicio es una estrategia muy consecuente y gratificante para perder grasa corporal y mejorar su salud.

El ejercicio como medio para quemar calorías ha sido reconocido por la industria del fitness. Hay muchos tipos de modalidades de ejercicio que se comercializan con el reclamo de "quemar más calorías", y el consumidor se pregunta qué es lo que determina la cantidad de calorías quemadas durante el ejercicio. Esta situación es la razón fundamental para escribir este artículo.
\end{abstract}

% Escriba el título en inglés.
\begin{englishtitle}
\noindent \textbf{Calorie burn prediction during exercise -}
\end{englishtitle}

%% Escriba el resumen en inglés.
\begin{abstractIng}
If people were to honestly answer the question 'What are the reasons you exercise?' a frequent answer would be to burn calories. In fact, according to the US Department of Health and Human Services (1992), 26 percent of American adults ages 20-74 are overweight, clearly demonstrating the impact of this national concern.

It is known that reducing body fat can reverse various disease processes (eg, type II diabetes, heart disease, etc.), exercise increases total caloric expenditure and also maximizes body fat loss and maintenance or gaining muscle mass, engaging in exercise is a very consistent and rewarding strategy for losing body fat and improving your health.

Exercise as a means of burning calories has been recognized by the fitness industry. There are many types of exercise modalities that are marketed as "burning more calories" and the consumer wonders what determines the number of calories burned during exercise. This situation is the fundamental reason for writing this article.
\end{abstractIng}

\end{frontmatter}

\begin{multicols}{2}   % JVS

%%
%% Start line numbering here if you want
%%
% \linenumbers

%% main text
\section{Introducción}
La principal directriz para la creación de este articulo es representar la solución a la interrogante:
\emph {¿Cual es el factor de mayor influencia en la quema de calorias durante el ejercicio? }  


Se evaluan diferentes variables que involucran las caracteristicas fisicas de la persona que realiza el ejercicio, contra otras variables, que describen la manera de hacer el ejercicio.
Las variables de estudio serán: Edad, Sexo, Peso, Estatura, Pulso cardiaco, Temperatura y Duración.

\\

{En reposo, el cuerpo gasta energía para mantener las funciones de las células que son esenciales para la vida. El bombeo continuo de sangre por parte del corazón exige energía, al igual que la ventilación continua (movimiento de aire hacia adentro y hacia afuera) de los pulmones. Además, mantener un entorno de soporte vital dentro y alrededor de las células requiere una descomposición constante de ciertas moléculas liberadoras de energía. Esta energía también se utiliza para formar las moléculas necesarias para reparar las células, almacenar energía (glucógeno y triglicéridos), combatir infecciones y procesar los nutrientes obtenidos de la digestión. Estas funciones exigentes de energía se combinan para formar la tasa metabólica basal del cuerpo, que puede variar de aproximadamente 800 a 1500 Kcal dependiendo del tamaño del cuerpo y la ingesta calórica total (cantidad ingerida de alimentos).}.


\subsection{Descomposición química}


El trifosfato de adenosina (ATP) es la molécula principal que el cuerpo utiliza como medio para utilizar la energía química para realizar el trabajo celular. El ejercicio aumenta el gasto calórico del cuerpo, ya que la contracción muscular implica la necesidad de formar y descomponer ATP repetidamente. La energía liberada por la descomposición del ATP alimenta la contracción del músculo esquelético, lo que aumenta las demandas de energía del cuerpo y aumenta el gasto calórico. Las investigaciones han demostrado que durante el ejercicio el aumento del gasto calórico se debe casi en su totalidad a la contracción del músculo esquelético; el equilibrio se debe a un aumento en las demandas de energía del corazón y los músculos utilizados durante la ventilación.


\begin{figure}[H]
\centering
  % El fichero es un eps y se convierte automáticamente a pdf con eps2pdf package
  \includegraphics[width=8cm]{Figuras/trifos}\\
  \caption{Estructura química del trifosfato.}\label{fig1}
\end{figure}

\section{Técnicas de Machine Learning}
Este documento hace una recopilación de conjuntos apropiados para enseñar a nuestros modelos de aprendizaje automático para que logre
saber cuál es la cantidad de calorías que el individuo gasta para quemar. De manera simultanea se divide el conjunto de datos en conjunto de entrenamiento y prueba. Aquí usamos SelectKBest, Kmeans y SVM como modelos de aprendizaje automático para comparar y luego evaluar estos modelos. La herramienta utilizada es Google Colaboratory o Google Colab es una herramienta basada en web y un servicio basado en la nube.

\subsection{Selección de caracteristicas}


La primera fase del análisis de caracterisiticas, se presentará atravéz de un ANOVA, esta herramienta es de gran utilidad, pues es una fórmula estadística que sirve para comparar las varianzas entre las medias (o el promedio) de diferentes grupos. Tambien se utiliza para determinar si existe alguna diferencia entre las medias de los diferentes grupos.

A continuación se presentan los resultados del valor de F, ordenando de las variables que tienen un valor más alto, al menos representativo.



\begin{table}[H]
  \caption{Resultados de ANOVA, F-Value por variable}
   \label{extremos45}
  \begin{tabular*}{\hsize}{lrrrrr}
\hline
Resultados de ANOVA\\
   \hline
Variable & F Value \\
Duration & 157053.43 \\
Heart Rate &  62387.94 \\
Body Temp & 31855.44 \\
Age	 &	18.904356\\
Weight & 366.25  \\
Gender & 7.5 \\
Height & 4.61 \\
\hline
  \end{tabular*}
\end{table}

En base a esta evaluación se  interpreta que un valor F alto, indica alta relación lineal; valores menores, lo contrario. Por lo tanto asumimos que existen 3 variables con mayor relación con la variable de respuesta {\emph {Burned Calories} y estas son \emph{ \emph {Duration, Heart Rate y Body Temp.}}


Se busca una segunda opción para la selección de caracteristicas, esto con la intención de revisar si existen resultados coincidentes, el metodo seleccionado es el de información mutua.

La información mutua mide la cantidad de información transferida cuando ${x}^{i}$= (variable de interes) es transmitido y ${y}^{i}$ =(variable de respuesta) es recibido.

\begin{figure}[H]
\centering
  % El fichero es un eps y se convierte automáticamente a pdf con eps2pdf package
  \includegraphics[height=3cm, width=8.5cm]{Figuras/in_mut}\\
  \caption{Título de la Figura 1. Recuerde que los títulos de figura que ocupan una sola línea van centrados. La figura es un fichero eps y gracias al paquete epstopdf se convierte automáticamente a pdf.}\label{fig1}
\end{figure}

Con el grafico anterior podemos observar que las variables Heart Rate, Weight y Body Temp comparten información mutua reelevante, en comparacion con el resto de las variables.



En base a lo anterior se cree que la eliminación de las variables Age y Gender pudiera ser conveniente para el agrupamiento de caracteristicas.

\subsection{Análisis de grupos}

En esta sección se realizará el análisis de grupos o tambien conocido como clustering, es la tarea de agrupar objetos por similitud, en conjuntos de manera que los miembros del mismo grupo tengan características similares. 

Es la tarea principal de la minería de datos exploratoria y es una técnica común en el análisis de datos estadísticos. Se puede realizar a travez del aprendizaje automatico No Supervisado y el Supervisado.

En primer instancia es importante mencionar que para esta sección del análisis, las variables ya fueron filtradas por medio de la selección de caracteristicas precedente, por lo que solo se tomarán en cuenta aquellas que aportan mayor información al modelo.

A continuación se mostraran los resultados del análisis por el metodo KMeans. Este algoritmo es bastante popular en su uso para la clasificación no supervisada (clusterización), inicialmente agrupa objetos en k grupos basándose en sus características. El agrupamiento se realiza minimizando la suma de distancias entre cada objeto y el centroide de su grupo o cluster. 

1) Defición de K y numero de centroides
Para definir la cantidad de clústers a aplicar, se efectuo la curva de codo. De manera lógica los primeros conglomerados agregarán mucha información, ya que los datos en realidad consisten en gran cantidad de grupos, pero una vez que el número de conglomerados exceda el número real de grupos en el data, la información agregada caerá bruscamente, porque solo está subdividiendo los grupos reales. Suponiendo que esto suceda, habrá un codo pronunciado en el gráfico de la variación explicada frente a los conglomerados, es el dato que se tomará para definir la cantidad de clústers.

\begin{figure}[H]
\centering
  % El fichero es un eps y se convierte automáticamente a pdf con eps2pdf package
  \includegraphics[height=4cm, width=7cm]{Figuras/elbow}\\
  \caption{Título de la Figura 1. Recuerde que los títulos de figura que ocupan una sola línea van centrados. La figura es un fichero eps y gracias al paquete epstopdf se convierte automáticamente a pdf.}\label{fig1}
\end{figure}

El codo de la curva decrece en 5, por lo que este será el número de clústers a utilizar.

\subsection{Creación del modelo mátematico}

Esta sección esta pendiente de completar \emph{Es importante que todas las fuentes estén incrustadas en el PDF resultante}.


\section{Conclusiones}

El objetivo de este artículo fue explicar la relación entre el ejercicio y el gasto calórico durante mediante la demostración de un estudio de caso. Si el compromiso con la actividad física relacionada con la salud es el objetivo para usted, estas son las recomendaciones:

PENDIENTE


Esperamos haberle proporcionado la información suficiente para que pueda tomar decisiones optimas en la realizacion de ejericio fisico. Cabe mencionar que los beneficios fisiológicos a largo plazo del ejercicio regular de la parte superior e inferior del cuerpo no se han dilucidado completamente en los resultados de la investigación.

%%%%%%%%%%%%%%%%%%%%%%%%%%%%%%%%%%%%%%%%%%%%%%%%%%%%%%%%%%%%%%%%%%%%%%%%%%%%%%%%%%%%%%%%%%

\section*{Agradecimientos}

Este trabajo ha sido realizado con el apoyo de los maestros de la Facultad de Ciencias Fisicomateticas de la UANL.


\section{Apendice}

\subsection{Más sobre figuras y tablas}

PENDIENTE


\section{Referencias}

PENDIENTE

\end{multicols}

\end{document}

%%
%% End of file `ejemplo latex RIAI.tex'.

